\chapter{Conclusão}
\epigraph{``\textit{O que vale na vida não é o ponto de partida e sim a caminhada. Caminhando e semeando, no fim terás o que colher}''.}{Cora Coralina}

\lettrine[lines=4, lhang=0.1, lraise=0, loversize=0.2, findent=0.1em]{\textcolor{corTema}{A}}{S}SIM finalizamos nosso livro! Vale a pena salientar que o que vimos durante o mesmo corresponde a uma pequena parte das funcionalidades disponíveis na linguagem de programação Java e no ``Mundo'' do desenvolvimento para Web, sendo que, para que você possa esgotar o assunto, será necessário o estudo de outras fontes, além de anos de prática com a linguagem em um ambiente de desenvolvimento real. 

Caso queira se especializar mais na linguagem Java, recomendo as obras:
\begin{itemize}
    \item \textcolor{blue}{\citetext{Deitel2017}}
    \item \textcolor{blue}{\citetext{HeadFirst2024}}
    \item \textcolor{blue}{\citetext{Horstmann2024a}}
    \item \textcolor{blue}{\citetext{Horstmann2024b}}
\end{itemize}

Em relação às tecnologias Web atuais, existem diversas alternativas para a resolução dos mais variados tipos de problemas. No ecossistema da plataforma Java, talvez um dos \textit{frameworks} \textit{back-end} mais famosos e utilizados seja o Spring Boot\footnote{\url{https://spring.io/projects/spring-boot}} que integra todas as soluções do \textit{framework} Spring de forma simplificada. Do ponto de vista de \textit{frameworks} \textit{front-end}, os mais famosos atualmente são o Angular\footnote{\url{https://angular.dev/}}, o React\footnote{\url{https://react.dev/}} e o Vue\footnote{\url{https://vuejs.org/}}.

Espero que este livro tenha sido útil!

Um grande abraço a todos!

Até mais!